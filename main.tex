% !TEX TS-program = pdflatex
% !TEX encoding = UTF-8 Unicode

\documentclass[11pt]{article} % use larger type; default would be 10pt

\usepackage[utf8]{inputenc} % set input encoding (not needed with XeLaTeX)

%%% PAGE DIMENSIONS
\usepackage{geometry} % to change the page dimensions
\geometry{a4paper} % or letterpaper (US) or a5paper or....
% \geometry{margin=2in} % for example, change the margins to 2 inches all round

\usepackage{graphicx} % support the \includegraphics command and options

%%% PACKAGES
\usepackage{booktabs} % for much better looking tables
\usepackage{array} % for better arrays (eg matrices) in maths
\usepackage{paralist} % very flexible & customisable lists (eg. enumerate/itemize, etc.)
\usepackage{verbatim} % adds environment for commenting out blocks of text & for better verbatim
\usepackage{subfig} % make it possible to include more than one captioned figure/table in a single float
\usepackage{amsmath} % More symbols
\usepackage{amsfonts} % More fonts
\usepackage{amsthm} % Theorems

%%% HEADERS & FOOTERS
\usepackage{fancyhdr} % This should be set AFTER setting up the page geometry
\pagestyle{fancy} % options: empty , plain , fancy
\renewcommand{\headrulewidth}{0pt} % customise the layout...
\lhead{}\chead{}\rhead{}
\lfoot{}\cfoot{\thepage}\rfoot{}

%%% SECTION TITLE APPEARANCE
\usepackage{sectsty}
\allsectionsfont{\sffamily\mdseries\upshape} % (See the fntguide.pdf for font help)
% (This matches ConTeXt defaults)

%%% ToC (table of contents) APPEARANCE
\usepackage[nottoc,notlof,notlot]{tocbibind} % Put the bibliography in the ToC
\usepackage[titles,subfigure]{tocloft} % Alter the style of the Table of Contents
\renewcommand{\cftsecfont}{\rmfamily\mdseries\upshape}
\renewcommand{\cftsecpagefont}{\rmfamily\mdseries\upshape} % No bold!

%%% Theorem stuff
\newtheorem{theorem}{Theorem}[section]
\newtheorem{corollary}[theorem]{Corollary}
\newtheorem{lemma}[theorem]{Lemma}

\theoremstyle{definition}
\newtheorem{definition}[theorem]{Definition}

\newtheorem{claim}[theorem]{Claim}

\newtheorem{example}[theorem]{Example}

\newcommand{\abs}[1]{\left|#1\right|}
\newcommand{\res}{\text{res}}

%%% END Article customizations

%%% The "real" document content comes below...

\title{Theory Of Functions Of A Complex Variable 2 - 2021B}
\author{Alon Ben-Tsur}
%\date{} % Activate to display a given date or no date (if empty),
         % otherwise the current date is printed 

\begin{document}
\maketitle

\section{Normal Convergence of Series and Products}

\subsection{Series and Elementary Examples}

We'll start with a definition:

\begin{definition}
Let $G \subseteq \mathbb{C}$ be an open set. A series of functions $f_n$ is said to be normally convergent (or locally normally convergnet, or compactly normally convergent) if for every compact set $K \subseteq G$, for some $n_0 = n\left(K\right)$, we have that:
\[ \sum _{n \geq n_0} \left\lVert f_n \right\rVert_{C\left(K\right)} = \sum _{n \geq n_0} \sup _K \left|f_n\right| < \infty \]
\end{definition}

Note, that we can make sense of this definition in the case of meromorphic functions (which might be "partially defined" on $G$, up to a discrete set). Noticing that if every point $z \in G$ has a neighborhood on which almost all $f_n$ functions are defined, and thus bounded on comapct sets, this definition holds.

It's clear to see that locally normally convergent series are also (locally) uniformly absolutely convergent, and thus we have the following corollary:

\begin{corollary}
If $f_n \in \mathcal{A}\left(G\right) = \mathcal{O}\left(G\right)$ are holomorphic functions (resp. $f_n \in \mathcal{M}\left(G\right)$ meromorphic functions) on $G$, such that $f = \sum_n f_n$ converges normally, then $f \in \mathcal{A}\left(G\right)$ (resp. $f \in \mathcal{M}\left(G\right)$).
\end{corollary}

\begin{example}
Consider the following examples of normally converging series, where $G = \mathbb{C}$:

\begin{enumerate}
\item $\left(\frac{\pi}{\sin\pi z}\right)^2 = \sum_{n\in\mathbb{Z}} \frac{1}{\left(z-n\right)^2}$
\item $\frac{\pi}{\sin\pi z}= -\frac{1}{z} + \sum_{n\in\mathbb{Z}\setminus\left\{0\right\}}\left(-1\right)^n \left(\frac{1}{z-n} + \frac{1}{n}\right)$
\item $\pi \cot \pi z =  \frac{1}{z} + \sum_{n\in\mathbb{Z}\setminus\left\{0\right\}} \left(\frac{1}{z-n} + \frac{1}{n}\right)$
\end{enumerate}
\end{example}

We will show two approaches one can take to prove these examples. The first approuch is as follows:

\begin{enumerate}
\item For the first example, show that the difference function is a bounded, entire function tending to $0$. Conclude that in that case, the difference must be identically $0$.
\item Taking deriviatives, deduce the third example from the first.
\end{enumerate}

\begin{proof}
Consider the difference function $g\left(x\right) = \left(\frac{\pi}{\sin\pi z}\right)^2 - \sum_{n\in\mathbb{Z}} \frac{1}{\left(z-n\right)^2}$.

We claim the $g$ is an entire function. Indeed, the only "suspected poles" of $g$ are at the integers $n \in \mathbb{Z}$. But the order of the pole of $\left(\frac{\pi}{\sin\pi z}\right)^2$ at $n$ is $2$, so subtracting $\frac{1}{\left(z-n\right)^2}$, we see that $g\left(n\right)$ is well defined (Intuitively, we "subtracted" all the poles).

Secondly, we claim that $g$ is $1$-periodic: $g\left(z+1\right) = g\left(z\right)$. Indeed, $\sin \pi \left(z + 1\right) = \sin \pi z$, and "shifting" $n$ forward, we see that $\sum_{n\in\mathbb{Z}} \frac{1}{\left(z+1-n\right)^2} = \sum_{n\in\mathbb{Z}}\frac{1}{\left(z-n\right)^2}$.

Thirdly, we have that $g\left(z\right) \to 0$ when $\abs{\Im z} \to \infty$.

Indeed, one can see that $\abs{\sin \pi z} \to \infty$ when $\abs{\Im z} \to \infty$ in multiple ways, for example considering $\sin\left(a + bi\right) = \sin a \cosh b + i \cos a \sinh b$: when $\abs{b} \to \infty$, both $\abs{\cosh b}$ and $\abs{\sinh b}$ go to infinity, and we can't have $\sin a$ and $\cos a$ go to $0$ at the same time, so at least of the componets (the real or the imaginary) explode to infinity.

Now, the series $\sum _{n\in\mathbb{Z}}\frac{1}{\left(z-n\right)^2}$ is actually globally uniformly convergent away from its poles (for example in $\left\{\abs{\Im z} > 1\right\}$), so we can exchange limits and see that that
\[ \lim _{\abs{\Im z} \to \infty}\sum _{n\in \mathbb{Z}} \frac{1}{\left(z-n\right)^2} = \sum_{n\in \mathbb{Z}} \lim_{\abs{\Im z} \to \infty} \frac{1}{\left(z-n\right)^2} = \sum_{n\in\mathbb{Z}} 0 = 0\]

Let us show that $g$ is bounded. Assume $\abs{g\left(z_n\right)} \to \infty$.

Considering $z_n - \left[\Re z_n\right]$, and the fact that $g\left(z+1\right) = g\left(z\right)$, we can assume that all $z_n$ are in the strip $\left\{0\leq \Re z \leq 1\right\}$.

So we must have that $\abs{\Im z_n} \to \infty$: Otherwise, $z_n$ is contained in a compact rectangle, and the limit of $\abs{g\left(z_n\right)}$ would be finite. But in that case, $\lim g\left(z_n\right) = 0$, so $\abs{g\left(z_n\right)}$ must be bounded, which is a contradiction.

We conclude that $g$ is a bouned, entire function. From Liouville's theorem, $g$ must be constant, and since it tends to $0$, we must have $g \equiv 0$.

We conclude that $\left(\frac{\pi}{\sin \pi z}\right)^2 = \sum_{n\in \mathbb{Z}} \frac{1}{\left(z-n\right)^2}$.

Now, differentiating term-by-term, we have that:
\[ -\frac{d}{dt}\left(\pi\cot \pi z\right) = \left(\frac{\pi}{\sin\pi z}\right)^2 = \sum_{n\in \mathbb{Z}} \frac{1}{\left(z-n\right)^2} = -\frac{d}{dt}\left(\frac{1}{z} + \sum_{n\in\mathbb{Z}\setminus\left\{0\right\}}\left(\frac{1}{z-n}+\frac{1}{n}\right)\right)\]

So $\pi \cot \pi z = \frac{1}{z} + \sum_{n\in\mathbb{Z}\setminus\left\{0\right\}}\left(\frac{1}{z-n}+\frac{1}{n}\right) + C$, where $C$ is some constant.

Now, both $\pi \cot \pi z$ and $\frac{1}{z} + \sum_{n\in\mathbb{Z}\setminus\left\{0\right\}}\left(\frac{1}{z-n}+\frac{1}{n}\right)$ are odd functions, so their difference, $C$, must be an odd function. But the only constant odd function is $0$, and we conclude that $\pi \cot \pi z = \frac{1}{z} + \sum_{n\in\mathbb{Z}\setminus\left\{0\right\}}\left(\frac{1}{z-n}+\frac{1}{n}\right)$.

For the last identity, we have that:
\[
\begin{split}
\frac{1}{z} + \sum_{n\in\mathbb{Z}\setminus\left\{0\right\}} \left(-1\right)^n \left(\frac{1}{z-n} + \frac{1}{n}\right) & = 2\left(\frac{1}{z} + \sum_{n\in\mathbb{Z}\setminus\left\{0\right\}}\left(\frac{1}{z-2n}+\frac{1}{2n}\right)\right) \\
& \qquad - \frac{1}{z} + \sum_{n\in\mathbb{Z}\setminus\left\{0\right\}}\left(\frac{1}{z-n}+\frac{1}{n}\right) \\
& = 2\cdot\frac{\pi}{2} \cot \frac{\pi z}{2} - \pi \cot \pi z \\
& = \frac{\pi}{\sin\pi z}
\end{split}
\]

\end{proof}

The second approach, due to Cauchy, is of the following fashion:

\begin{enumerate}
\item Show either the second or the third example using contour integration.
\item Deduce the first example from the third taking deriviatives.
\end{enumerate}

\begin{proof}

Set $f\left(z\right)$ to be either $\pi \cot \pi z$ or $\frac{\pi}{\sin \pi z}$.

Fix some $z$.

Denote the square $Q_N = \left\{\abs{\Re z}, \abs{\Im z} \leq N + \frac{1}{2} \right\}$. Consider its perimeter $\partial Q_N$, going counter clockwise, and set $I_n = \frac{1}{2\pi i}\int_{\partial Q_N} f\left(\xi\right)\left(\frac{1}{\xi - z} - \frac{1}{\xi}\right)d \xi$.

Since $f$ is bounded on all the squares $Q_N$, and the integral of the $\frac{1}{\xi -z}, \frac{1}{z}$ parts is inverse quadratic, we see that:
\[ \abs{I_N} \leq \frac{C}{N^2} \cdot\text{Length} \left(\partial Q_N \right) \cdot \max_{\partial Q_N} \abs{f} \to 0\]

From the residue theorem, we have that $I_N = \sum_{\abs{n} \leq N} \res_n f - f\left(z\right)$. Taking $n \to \infty$, we obtain our results.
\end{proof}

\subsection{Products and Elementary Examples}

\begin{definition}
Suppose that $f_n = 1 + a_n$, that is, $a_n = f_n - 1$. We say that $\prod_n f_n$ normally converges in $G$ if $\sum _n a_n$ normally converges in $G$.    
\end{definition}

\begin{claim}
The following properties of normally convergent products hold:

\begin{enumerate}
\item Rearrangement of order: given a permutation $\sigma$ of the natural numbers, we have that $\prod_n f_n = \prod_n f_{\sigma\left(n\right)}$.
\item Products of analytic/meromorphic functions: $\left(f_n\right) \subseteq \mathcal{A}\left(G\right)$ (resp. $\mathcal{M}\left(G\right)$) implies that $\prod_n f_n \in \mathcal{A}\left(G\right)$ (resp. $\mathcal{M}\left(G\right)$).
\item If we denote the zero set of a function by $Z$, we have that $Z\left(\prod_n f_n\right) = \bigcup_n Z\left(f_n\right)$.
\item Normal convergence of logarithmic derivative: $\sum_n \frac{f_n'}{f_n}$ normally converges.
\end{enumerate}
\end{claim}

The proofs of these properties are left as an exercise (I'll probably add them later).

\begin{example}
We have that:
\[ \sin \pi z = \pi z \prod_{n\neq 0}\left(1 - \frac{z}{n}\right)e^{\frac{z}{n}} = \pi z \prod _{n\geq 1} \left(1 - \frac{z^2}{n^2}\right)\]
\end{example}

\begin{proof}
Consider the logarithmic derivative:
\[ \pi \cot \pi z = \frac{1}{z} + \sum _{n\in \mathbb{Z}\setminus\left\{0\right\}}\left(\frac{1}{z-n} + \frac{1}{n}\right) = \frac{1}{z} + \sum_{n\in \mathbb{Z}\setminus\left\{0\right\}} \left(\log \left(1- \frac{z}{n}\right) + \frac{z}{n}\right)'\]

So we have that $\sin \pi z = C z \prod_{n\neq 0}\left(1 - \frac{z}{n}\right)e^{\frac{z}{n}}$. Dividing by $z$ and taking $z \to 0$, we see that $C = \pi$.
\end{proof}

Here are another few examples we will see during the course:

\begin{example}
\begin{enumerate}
\item $\prod_{n=0}^\infty \left(1 + z^{2^m}\right) = \frac{1}{1-z}$, where $\abs{z} < 1$.
\begin{item}
$\prod_{n=0}^\infty \left(1 - z^{2^m}\right) = \sum_{n\geq 0} t\left(n\right) z^n$, where $\abs{z} < $.

Here $t\left(n\right)$ is the Thue-Morse sequence: $t\left(n\right) = \left(-1\right)^{\omega\left(n\right)}$, where $\omega\left(n\right)$ is the number of $1$'s in the binary expansion of $n$.

There is a simple recurrence relation: $t\left(n\right) = 0$, $t\left(2n\right) = t\left(n\right)$, $t\left(2n+1\right) = -t\left(n\right)$.
\end{item}

\begin{item}
$\prod_{m=1}^\infty \frac{1}{1- z^m} = \sum_{n \geq 0}p\left(n\right)z^n$, where $\abs{z} < 1$.

Here, $p\left(n\right)$ is the partition function: $p\left(n\right)$ is the number of ways we can write $n$ as a sum of natural ($\geq 0$) numbers, disregarding order.

There is a theorem of Hardy and Ramanujan: $p\left(n\right) \approx \frac{1}{4n\sqrt{3}} e^{\pi\sqrt{\frac{2n}{3}}}$.
\end{item}

\item $\prod_p \left(1-\frac{1}{p^s}\right)^{-1} = \sum_{n \geq 1} \frac{1}{n^s}$, where $\Re s > 1$.
\end{enumerate}
\end{example}

\section{Euler Gamma Function}

\subsection{Definition and Basic Properties}

\begin{definition}
The Euler Gamma Function is defined on $\left\{\Re z > 0\right\}$ by the following formula:
\[ \Gamma \left(z\right) = \int_0^\infty t^{z-1}e^{-t}dt \]
\end{definition}

\begin{claim}
The following functional equation holds:
\[ \Gamma \left(z+1\right) = z \Gamma\left(z\right) \]
\end{claim}

\begin{proof}
Integrating by parts, setting $u = t^z$, $v' = e^{-t}$, $u' = zt^{z-1}$, $v = -e^{-t}$, we have that:

\[
\begin{split}
\Gamma\left(z+1\right) = & \int_0^\infty t^z e^{-t}dt \\
= & \left[-zt^{z-1}e^{-t}\right]_0^\infty + z\int_0^\infty t^{z-1}e^{-t}dt \\
= & 0 + z\Gamma\left(z\right)
\end{split}
\]
\end{proof}

Now, we would like to show that $\Gamma \in \mathcal{A}\left(\Re z > 1\right)$. We'll need some machinery concerning integrals of analytic functions.

\begin{lemma}
Let $F : G \times \left[0,1\right] \to \mathbb{C}$ be a continuous function, such that for all $t \in \left[0,1\right]$, the function $f\left(z\right) = F\left(z,t\right)$ is analytic on $G$. Then the definite integral $z \mapsto \int_0^1 F\left(z,t\right)dt$ is analytic on $G$.
\end{lemma}

\begin{proof}
The trick is to approximate the integral using Riemann sums.

Set $f_n \left(z\right) = \frac{1}{n} \cdot \sum_{k=0}^{n-1} F\left(z, \frac{k}{n}\right)$. We have that $f_n \to f$ locally uniformly.

Indeed, fix some $\varepsilon > 0$, and some $z \in G$. Take a closed disk $\bar{D} \subseteq G$ around $z$ and some $\delta > 0$, such that if $\abs{t-t'} < \delta$ we have that $\sup_{w\in\bar{D}}\abs{F\left(w, t\right) - F\left(w, t'\right)} < \varepsilon$. Then for $n > \frac{1}{\delta}$ we have that:
\[
\begin{split}
\sup_{w \in \bar{D}} \abs{f - f_n} & = \sup_{w\in\bar{D}}\abs{\int_0^1 F\left(w,t\right)dt - \frac{1}{n}\sum_{k=0}^{n-1} F\left(w, \frac{k}{n}\right)} \\
& \leq \sup_{w\in\bar{D}}\sum_{k=0}^{n-1}\int_{\frac{k}{n}}^{\frac{k+1}{n}} \abs{F\left(w,t\right) - F\left(w, \frac{k}{n}\right)}dt \\
& \leq \sum_{k=0}^{n-1}\frac{1}{n}\varepsilon \\
& = \varepsilon
\end{split}
\]

Since each $f_n$ is analytic, we conclude that $f$ is analytic.
\end{proof}

\begin{claim}
$\Gamma \in \mathcal{A}\left(\Re z > 0 \right)$.
\end{claim}

\begin{proof}
Set $f_n\left(z\right) = \int_{\frac{1}{n}}^n t^{z-1}e^{-t}dt$. From the last lemma, each $f_n$ is analytic. We will show that $f_n$ converges to $\Gamma$ locally uniformly.

Fix some $z_0$ with $\Re z > 0$.

We'll start with approximating $\int_0^{\frac{1}{n}} t^{z-1} e^{-t} dt$.

We have that $1 \geq e^{-t}$ for all $t\in \left[0,\frac{1}{n}\right]$. Take some $\Re z_0 > \delta > 0$. We have that $\abs{t^{z-1}} < t^{\delta - 1}$ for all $t\in \left[0,\frac{1}{n}\right]$. We conclude that:
\[ \abs{\int_0^{\frac{1}{n}}t^{z-1} e^{-t} dt} \leq \int_0^{\frac{1}{n}} t^{\delta - 1} dt = \delta^{-1} n^{-\delta} \to 0\]
that is, in $\Re z_0 > \delta$, we have that $\int_0^{\frac{1}{n}}t^{z-1}e^{-t} dt$ converges uniformly to $0$.

For the other tail, take some $\Re z_0 < M$. For all $t > 1$, we have that $\abs{t^{z - 1}} \leq t^{M - 1}$. Now, consider $t^{M-1}e^{-\frac{t}{2}}$. This goes to $0$ as $t \to \infty$, so at some point, for large enough $t$, $t^{M-1} \leq e^{\frac{t}{2}}$. Equivalently, we have that $t^{M-1}e^{-t} \leq e^{-\frac{t}{2}}$. We conclude that for large enough $n$:
\[ \abs{\int _n ^\infty t^{z-1}e^{-t} dt} \leq \int_n ^\infty e^{-\frac{t}{2}}dt \to 0 \]
that is, in $\Re z_0 < M$, we have that $\int_n^\infty t^{z-1}e^{-t}dt$ converges uniformly to $0$.

Finally, the function $\Gamma - f_n$ converges locally uniformly to $0$, so $f_n$ converges locally uniformly to $\Gamma$ and so $\Gamma$ is analytic.
\end{proof}

\begin{claim}
$\Gamma$ can be extended to a meromorphic function on the entire plane. Furthermore, the only poles of $\Gamma$ are the non-positive integers, and they are simple poles with residues $\res_{-n} \Gamma = \frac{\left(-1\right)^n}{n!}$.
\end{claim}

We will sketch two proofs for this fact, one of which relies on the functional equation $\Gamma$ satisfies, and the other one relies on the definition of $\Gamma$ using integrals.

\begin{proof}
Consider some $z$ with $\Re z > -1$. The expression $\Gamma\left(z\right) = \frac{\Gamma\left(z+1\right)}{z}$ is well defined (except maybe at $z = 0$). It clearly defines a holomorphic function (in its domain of definition, that is, without $0$), which extends the $\Gamma$ function.

Continuing with this fashion, we can set for $z$ with $\Re z > -2$: $\Gamma\left(z\right) = \frac{\Gamma\left(z+2\right)}{z\left(z+1\right)}$, which defines a holomorhpic function on its domain of definition.

Each step, for $\Re z > n$, we have that $\Gamma\left(z\right) = \frac{\Gamma\left(z + n + 1\right)}{z\left(z+1\right)\dots\left(z + n\right)}$.

Finally, we get an extension of $\Gamma$, holomorphic on the entire plane without the nonpositive integers.

Computing residues, we have that $\res_0 \Gamma = \Gamma\left(1\right) = 1$, $\res_{-1}\Gamma = \frac{\Gamma\left(1\right)}{-1} = -1$, and so on, $\res_{-n}\Gamma= \frac{\Gamma\left(1\right)}{\left(-n\right)\left(-n+1\right)\dots\left(-n+n-1\right)} = \frac{\left(-1\right)^n}{n!}$.

We conclude that $\Gamma$ can be extended to a meromorphic satisfying the claim.
\end{proof}

Now, for the proof using the integral definition:

\begin{proof}
Consider $\Gamma\left(z\right) = \int_0 ^1 t^{z-1}e^{-t}dt + \int_1^\infty t^{z-1}e^{-t}dt$.

We saw in our proof that $\Gamma$ is analytic that the left integral is actually locally uniformly convergnet on the entire plane, that is, it defines an entire function $g\left(z\right)$. All that is left to show is that the right integral can be extended to a meromorphic function on the entire plane with the desired residues.

Fixing some $\Re z \geq \delta >0$, we uniform convergence, so we can exchange the series defining $e^{-t}$ and the integral:
\[ \int_0 ^1 t^{z-1} e^{-t} dt = \int _0 ^1 t^{z-1} \sum_{n=0}^\infty \frac{\left(-1\right)^n}{n!} t^n dt = \sum_{n=0}^\infty \frac{\left(-1\right)^n}{n!} \int_0^1 t^{z+n-1}dt = \sum_{n=0}^\infty \frac{\left(-1\right)^n}{n!}\frac{1}{n+z} \]

But the left hand series actually locally uniformly converges on the entire plane to a meromorphic with the desired simple poles and residues, so the $\Gamma$ function can be extended to such a function.
\end{proof}

The technique displayed at the last proof, of taking a function defined by an integral or a series with a limited domain of convergence, and rewriting them using integration by parts or other sorts of expansion to extend their domain of definitions, will also be used later in the course.

Consider $\frac{1}{\Gamma}$. It is a meromorphic function defined on the entire plane. We will show that it is actually entire, using the following formula.

\begin{claim}
We have that:
\[ \frac{1}{\Gamma\left(z\right)} = ze^{\gamma z} \prod_{z \geq 1} \left(1+\frac{z}{n}\right)e^{-\frac{z}{n}}\]

where $\gamma = \lim_{N\to \infty} \left(\sum_{n=1}^N\frac{1}{n} - \log N\right)$ is the Euler-Mascheroni constant.  
\end{claim}

\begin{proof}
Consider the functions
\[
f_n\left(t\right) =
\begin{cases}
\left(1-\frac{t}{n}\right)^n, & 0\leq t \leq n \\
0, & t > n
\end{cases}
\]

From basic calculus, we know that $0 \leq f_n \leq e^{-t}$, and that $f_n \uparrow e^{-t}$.

Fix some $z$ with $\Re z > 0$. We have that:
\[ \int _0^\infty f_n\left(t\right) t^{z-1} dt = \int_0^n \left(1- \frac{t}{n}\right)^n t^{z-1}dt \]

Integrating by parts, setting $u = \left(1 - \frac{t}{n}\right)^n$, $v' = t^{z-1}$, $u' = -\frac{n}{n}\left(1-\frac{t}{n}\right)^{n-1}$, $v = \frac{1}{z}t^{z+1-1}$, we have that:
\[
\begin{split}    
\int_0^n \left(1-\frac{t}{n}\right)^n t^{z-1} dt & = \left[\frac{1}{z}\left(1-\frac{t}{n}\right)^nt^{z+1 -1}\right]_0^n + \frac{n}{n}\cdot\frac{1}{z}\int_0^n \left(1-\frac{t}{n}\right)^{n-1}t^{z+1-1}dt \\
& = 0 + \frac{n}{n}\cdot\frac{1}{z}\int_0^n \left(1-\frac{t}{n}\right)^{n-1}t^{z+1-1}dt \\
& = \frac{n}{n}\cdot\frac{1}{z}\int_0^n \left(1-\frac{t}{n}\right)^{n-1}t^{z+1-1}dt
\end{split}
\]

Integrating by parts again, this time with $u = \left(1 + \frac{t}{n}\right)^{n-1}$, $v' = t^{z+1-1}$, we have that:
\[ \int_0^n \left(1-\frac{t}{n}\right)^n t^{z-1} dt = \frac{n\left(n-1\right)}{n^2}\cdot\frac{1}{z\left(z+1\right)}\int_0^n \left(1-\frac{t}{n}\right)^{n-2}t^{z+2-1}dt \]

Continuing like this $n$-times, we finally have that:
\[
\begin{split}
\int_0^n \left(1-\frac{t}{n}\right)^n t^{z-1} dt &= \frac{n!}{n^n} \cdot \frac{1}{z\left(z+1\right)\cdots\left(z+n-1\right)}\int_0^n t^{z+n-1}dt \\
& = \frac{n!}{n^n} \cdot \frac{n^{z+n}}{z\left(z+1\right)\cdots\left(z+n\right)}
\end{split}
\]

Interchanging the order of the improper integral and limit (TODO: why can we do that?), we have that:
\[ \Gamma\left(z\right) = \lim_{n\to\infty} \frac{n!}{n^n} \frac{n^{z+n}}{z\left(z+1\right)\cdots\left(z+n\right)} \] 
or equivalently:
\[ \frac{1}{\Gamma\left(z\right)} = \lim_{n\to \infty} \frac{n^n}{n!}\frac{z\left(z+1\right)\cdots\left(z+n\right)}{n^{z+n}} \]

Rearranging, we have that:
\[ \frac{n^n}{n!}\frac{z\left(z+1\right)\cdots\left(z+n\right)}{n^{z+n}} = z n^{-z} \left(1 + \frac{z}{1}\right)\left(1 + \frac{z}{2}\right)\dots\left(1 + \frac{z}{n}\right)\]
and clearly:
\[n^{-z} = e^{-z\log n} = e^{-z\left(\log n - \sum_{k=1}^n\frac{1}{k}\right)} \cdot e^{-\frac{z}{1}} \cdot e^{-\frac{z}{2}}\cdot \dots \cdot e^{-\frac{1}{n}}\]
and finally, for all $z$ with $\Re z > 0$, we have that:
\[ \frac{1}{\Gamma\left(z\right)} = ze^{\gamma z}\prod_{n\geq1}\left(1+\frac{z}{n}\right)e^{-\frac{z}{n}}\]

From the identity theorem, the equality holds in the entire complex plane.
\end{proof}

Now, from the product, it is clear that $\frac{1}{\Gamma}$ is entire. This means that $\Gamma$ has no zeroes in the entire plane.

Now, for the famous Euler Reflection Formula:

\begin{claim}
For all $z \notin \mathbb{Z}$ we have that $\Gamma\left(z\right)\Gamma\left(1-z\right) = \frac{\pi}{\sin\pi z}$.
\end{claim}

\begin{proof}
We have that:
\[
\frac{1}{\Gamma\left(z\right)\Gamma\left(-z\right)} = -z^2 \prod_{n\geq 1}\left(1 - \frac{z^2}{n^2}\right) = -\frac{z}{\pi}\sin\pi z
\]
so:
\[ \Gamma\left(z\right)\Gamma\left(1-z\right) = -z\Gamma\left(z\right)\Gamma\left(-z\right) = \frac{\pi}{\sin\pi z} \]
\end{proof}

Finally, we get our first non-trivial value of the Gamma function:

\begin{corollary}
$\Gamma\left(\frac{1}{2}\right) = \sqrt{\pi}$.
\end{corollary}

Another corollary of the expansion of $\frac{1}{\Gamma}$ as an infinte product, is an expression of the logarithmic derivative of $\Gamma$ as a series:

\begin{corollary}
The logarithmic derivative of the Gamma function satisfies: $\frac{\Gamma'}{\Gamma} = -\gamma - \frac{1}{z} - \sum_{n\geq 1}\left(\frac{1}{z+n} - \frac{1}{n}\right)$. Thus, the second logarithmic derivative of the Gamma Function satisfies $\left(\frac{\Gamma'}{\Gamma}\right)' = \sum _{n \geq 0}\frac{1}{\left(z+n\right)^2}$.
\end{corollary}

We conclude the subsection by listing a few of the properties of the Gamma function we showed:

\begin{enumerate}
\item $\Gamma\left(z+1\right) = z\Gamma\left(z\right)$.
\item $\Gamma \in \mathcal{A}\left(\Re z > 0\right)$.
\item $\Gamma$ has a meromorphic continuation to the entire plane, with simple poles in the points $0,-1,-2,\dots$ such that $\res_{-n} \Gamma = \frac{\left(-1\right)^n}{n!}$.
\item $\frac{1}{\Gamma\left(z\right)} = ze^{\gamma z}\prod_{n \geq 1}\left(1+\frac{z}{n}\right)e^{-\frac{z}{n}}$.
\item $\Gamma\left(z\right)\Gamma\left(1-z\right) = \frac{\pi}{\sin \pi z}$.
\end{enumerate}

\end{document}

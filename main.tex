% !TEX TS-program = pdflatex
% !TEX encoding = UTF-8 Unicode

\documentclass[11pt]{article} % use larger type; default would be 10pt

\usepackage[utf8]{inputenc} % set input encoding (not needed with XeLaTeX)

%%% PAGE DIMENSIONS
\usepackage{geometry} % to change the page dimensions
\geometry{a4paper} % or letterpaper (US) or a5paper or....
% \geometry{margin=2in} % for example, change the margins to 2 inches all round

\usepackage{graphicx} % support the \includegraphics command and options

%%% PACKAGES
\usepackage{booktabs} % for much better looking tables
\usepackage{array} % for better arrays (eg matrices) in maths
\usepackage{paralist} % very flexible & customisable lists (eg. enumerate/itemize, etc.)
\usepackage{verbatim} % adds environment for commenting out blocks of text & for better verbatim
\usepackage{subfig} % make it possible to include more than one captioned figure/table in a single float
\usepackage{amsmath} % More symbols
\usepackage{amsfonts} % More fonts
\usepackage{amsthm} % Theorems

%%% HEADERS & FOOTERS
\usepackage{fancyhdr} % This should be set AFTER setting up the page geometry
\pagestyle{fancy} % options: empty , plain , fancy
\renewcommand{\headrulewidth}{0pt} % customise the layout...
\lhead{}\chead{}\rhead{}
\lfoot{}\cfoot{\thepage}\rfoot{}

%%% SECTION TITLE APPEARANCE
\usepackage{sectsty}
\allsectionsfont{\sffamily\mdseries\upshape} % (See the fntguide.pdf for font help)
% (This matches ConTeXt defaults)

%%% ToC (table of contents) APPEARANCE
\usepackage[nottoc,notlof,notlot]{tocbibind} % Put the bibliography in the ToC
\usepackage[titles,subfigure]{tocloft} % Alter the style of the Table of Contents
\renewcommand{\cftsecfont}{\rmfamily\mdseries\upshape}
\renewcommand{\cftsecpagefont}{\rmfamily\mdseries\upshape} % No bold!

%%% Theorem stuff
\newtheorem{theorem}{Theorem}[section]
\newtheorem{corollary}[theorem]{Corollary}

\theoremstyle{definition}
\newtheorem{definition}[theorem]{Definition}

\newtheorem{example}[theorem]{Example}

%%% END Article customizations

%%% The "real" document content comes below...

\title{Theory Of Functions Of A Complex Variable 2 - 2021B}
\author{Alon Ben-Tsur}
%\date{} % Activate to display a given date or no date (if empty),
         % otherwise the current date is printed 

\begin{document}
\maketitle

\section{Normal Convergence of Series and Products}

We'll start with a definition:

\begin{definition}
Let $G \subseteq \mathbb{C}$ be an open set. A series of functions $f_n$ is said to be normally convergent (or locally normally convergnet, or compactly normally convergent) if for every compact set $K \subseteq G$, for some $n_0 = n\left(K\right)$, we have that:
\[ \sum _{n \geq n_0} \left\lVert f_n \right\rVert_{C\left(K\right)} = \sum _{n \geq n_0} \sup _K \left|f_n\right| < \infty \]
\end{definition}

Note, that we can make sense of this definition in the case of meromorphic functions (which might be "partially defined" on $G$, up to a discrete set). Noticing that if every point $z \in G$ has a neighborhood on which almost all $f_n$ functions are defined, and thus bounded on comapct sets, this definition holds.

It's clear to see that locally normally convergent series are also (locally) uniformly absolutely convergent, and thus we have the following corollary:

\begin{corollary}
If $f_n \in \mathcal{A}\left(G\right) = \mathcal{O}\left(G\right)$ are holomorphic functions (resp. $f_n \in \mathcal{M}\left(G\right)$ mermorphic functions) on $G$, such that $f = \sum_n f_n$ converges normally, then $f \in \mathcal{A}\left(G\right)$ (resp. $f \in \mathcal{M}\left(G\right)$).
\end{corollary}

\begin{example}
Consider the following examples of normally converging series:

\begin{enumerate}
\item $\left(\frac{\pi}{\sin\pi z}\right)^2 = \sum_{n\in\mathbb{Z}} \frac{1}{\left(z-n\right)^2}$
\item $\frac{\pi}{\sin\pi z}= -\frac{1}{z} + \sum_{n\in\mathbb{Z}\setminus\left\{0\right\}}\left(-1\right)^n \left(\frac{1}{z-n} + \frac{1}{n}\right)$
\item $\pi \cot \pi z =  \frac{1}{z} + \sum_{n\in\mathbb{Z}\setminus\left\{0\right\}} \left(\frac{1}{z-n} + \frac{1}{n}\right)$
\end{enumerate}
\end{example}

Let us prove the first example.

\subsection{A subsection}

More text.

\end{document}
